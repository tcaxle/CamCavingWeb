\documentclass[a4paper,11pt]{article}

\usepackage{a4wide}
\usepackage{charter}
\usepackage[dvips]{graphicx}

\DeclareFixedFont{\titlefont}{OT1}{phv}{b}{n}{16pt}
\DeclareFixedFont{\titlefontsmall}{OT1}{phv}{m}{n}{14pt}
\DeclareFixedFont{\headingfont}{OT1}{phv}{b}{n}{12pt}
\DeclareFixedFont{\subheadingfont}{OT1}{phv}{b}{n}{11pt}
\DeclareFixedFont{\subsubheadingfont}{OT1}{phv}{m}{n}{11pt}

\newcommand{\heading}[1]{\vspace*{8pt}{\headingfont #1}}
\newcommand{\subheading}[1]{\vspace*{8pt}{\subheadingfont #1}}
\newcommand{\subsubheading}[1]{\vspace*{8pt}{\subsubheadingfont #1}}

\setlength{\parskip}{1.5ex}
\setlength{\parindent}{0pt}

\begin{document}

{\titlefontsmall Cambridge University Caving Club}\smallskip\\
{\titlefont A Guide for Novice Caving}\medskip\\

If you have decided or are wondering about going caving with CUCC, then the
following information should be of some help.  We find caving exciting and
interesting; it is a completely new experience even for those that are used to
climbing or other outdoor sports.

Cambridge happens to be one of the flattest areas you can find yourself in
(just in case you had not noticed yet); thus CUCC caving trips in Britain will
primarily take place in the Yorkshire Dales, Peak District, South Wales and in
the Mendips.  Half way through term you'll be desperate to get out of Cambridge
and these areas provide a perfect change of scenery.  Throughout the year
weekend trips (Friday to Sunday night) will take place in these areas
and by the end of the year you should be acquainted with most of these caving
``hot spots''.

As the year progresses the caving trips will become longer and more technical,
which is why coming on an introductory meet is strongly recommended to gain
experience.  It's also an excellent time to meet people. The first novice trip
will actually be half a weekend (Saturday evening to Sunday night) and should
give an excellent taster of what is to come.

Although previous experience in terms of climbing or even caving might be
helpful, it is not required and complete beginners are very welcome.  It is,
however, important that adults who seek to go caving are aware of, and accept
the element of risk and therefore take responsibility for their own actions,
not simply relying on more experienced members.  It should be said at this point
that caving (and CUCC in particular) does have a very good safety record.

What follows are some practical guidelines which will help you prepare for a
weekend and help us ensure your safety.

\subsection*{Meeting up\ldots}

Basic details such as when and where trips are occurring will be present on the
CUCC website ({\tt http://www.srcf.ucam.org/caving/}).  More detailed information will
be presented at the weekly pub meets.  These occur on Tuesdays at the Castle
Inn (on Castle Street, up from Magdalene Bridge) from 9pm onwards.  Coming
to the pubmeets is highly encouraged as more detailed plans for the following
weekend will be discussed.

Once registered on the e-mail lists you will receive all the current and up to
date information relating to departure times and transport allocations.

The pub meet is also a good time to bring along cheques, made out to
``Cambridge University Caving Club'',  for the weekends.  Costs range around
\pounds 25-\pounds 30, most of which is spent on providing transport.
Alternatively money can be sent via ICMS (inter-collegiate mail service---this
is free) to Serena Povia at Churchill College.

\newpage
\subsection*{Setting off\ldots}

Regardless of whether we are planning to set off on Friday or Saturday night,
we will meet beforehand outside the tackle store, which is the garage to the left of 6 Grange Road. You will have received a sheet with directions to the Tackle Store along with this document.

Usually we will meet up sometime between 5.00 and 6.30pm.  However, please try
to be there at the time planned beforehand in the pub as it isn't fun for
anyone stuck waiting.

\subsection*{Equipment and safety-critical tackle}

At the Tackle Store you will be issued with all the equipment you will need.
This can be tried on to see if it fits, but it might be useful to know things
like your shoe size in advance.

You'll be given:

\begin{itemize}
\item A `furry' or fleece undersuit
\item An oversuit
\item Welly boots
\item Wide cell belt (formerly called a belay belt)
\item Helmet
\item Light
\item Knee pads
\item A bag to fit it all in
\end{itemize}

Depending on the type of trip you might also be issued with:

\begin{itemize}
\item Harness
\item SRT (ropework) kit
\end{itemize}

{\bf Note:} we will expect you to wash your gear, especially the undersuit and
oversuit, after the weekend.  Equipment such as wellies and the
kneepads just need to be rinsed e.g. with a hose or in the bath.  This needn't
take long if you don't leave it mouldering in coagulated mud for a week :-)

\newpage

\subsection*{Personal gear / stuff to bring}

\begin{itemize}

\item Swimwear / old underwear (``shreddies'') -- so named because caving destroys
them and colours them nicely in irremovable brown stains. 

\item Walking socks / wetsuit socks for wearing underground.

\item High energy bars for in the cave: Mars bars, Snickers and the like are
particularly popular to give that extra boost of sugar.

\item T-shirt -- some argue this is more comfortable / warmer under the main caving
suit.  Alternatively thermal underwear might also be a good substitute.

\item Sleeping bag -- a 2-3 season one is recommended: some of the caving huts can be a
bit cold in the winter.

\item Towel / usual toiletry items.

\item Waterproof coat and decent footwear -- it can often be quite muddy/wet even
when not in the cave if the weather is bad.

\item Washing up gloves for caving -- some claim this is warmer, while
others swear the opposite and don't wear any.  Regardless, they might prevent rocks
scratching your hands too much.

\item A spare set of warm clothes including fleece/jumper etc.

\item Hiking gear -- you might like to take a day off from caving.  The
areas in which the caves are located are ideal for walking trips.
\end{itemize}

The club provides food, but we might stop by for a pint at a pub after the
caving trips (to chat about all the day's ordeals!).  On the drive up or back
we tend to stop for a curry or chips, so you should bring some money for this.

\subsection*{The fun bit\ldots CAVING!}

\subsection*{The team}

On introductory trips a trip leader will be appointed. He/she will be of
sufficient experience to lead the trip and must be reasonably confident of the
abilities and experience of the members of the party to safely complete the
trip, as well as being confident that adequate tackle has been taken.
Normally a seconder will also be chosen who is also able to lead the trip
should it be necessary.

Just like you, they are here to have fun and any concerns or questions should be
directed to them as soon as possible.  Trip length and difficulty are geared to
suit the individuals going on the trip so there should be no need to worry and
only time to enjoy oneself.

CUCC sees formal leadership systems as undesirable in
recreational caving conducted by adults. The essence of caving should be
individual competence and companionship of friends, unfettered by unnecessary
rules and regulations.  Therefore on subsequent trips no-one in particular is
designated as the leader.  It is up to everyone on a trip to make sure that
they are all up to it, and to look out for each other.  Participants should not
be afraid to make the more experienced members of the group aware of any problems
they may perceive.  Caves should be chosen to be suitable for all the
participants.

\subsection*{Safety}

\subsubsection*{The caving gear and you}

\begin{itemize}
\item Check that your helmet, wide cell belt and harness (if you have one) are in
good condition.  Please {\bf ask} us if you have any queries.  In
the interest of safety, please make sure you harness is adjusted to your size,
and that the buckles are doubled back.

\item Check that your light turns on and off on the main and pilot settings.

\item Check that your boots fit comfortably.

\item Your suit will feel
tight the first time you wear it, but this should improve whilst caving.  If
you are still unsure about the fit, ask sooner rather than later.
\end{itemize}

\begin{quote}
``The first time I tried to put on my caving gear, I felt like I didn't know my
left from my right and everything was a jumble.  It can be confusing first time
round but I assure you that everyone has had the same problem and we will help
you put it on and explain what goes where.''
\end{quote}

\subsubsection*{In the cave}

Prior to the trip, a brief outline of what is to be expected will be given.
Novice trips tend to last between 2-6 hours, and subsequent trips rarely last
more than eight hours.  If at any time during this period you decide that this
is not for you, or you do not want to do a climb down etc., then {\bf say so}.

Caves can be intimidating environments first time round and we prefer it if you
tell us immediately if you would like to go back.  Nobody will laugh or think
you odd.

Caving has a very low accident rate and cave rescues are rare, but difficult.
Even something simple like a sprained ankle can add hours to a cave trip as it
is not an easy environment to move around and operate in.  Thus if you are
feeling tired, or ill, or simply want to go back {\bf tell us}, we don't bite!

Remember, we are going caving to have a good time and some fun: it is not a
competitive sport.  You will be told of potential hazards e.g. squeezes, water
hazards, false floors, pitches (vertical drops) and climbs.

Please do follow instructions carefully
as the leader will understand the situation and the necessary response.  But
remember: caving does have an element of risk, and the more experienced members
of the group will not be able to warn you of every hazard, so pay attention and
check that what you are doing seems sensible.

All relevant techniques related to ascending or descending pitches will be
explained on the trip if not before and the team leaders will help you with
attaching karabiners (krabs), harnesses and other necessary equipment.
Normally, above ground training such as ladder and rope (SRT) techniques, will
be practiced before hand.

Caves are generally
wet and cold environments and there may be a slight risk of hypothermia.
Another possible hazard in some caves is caused by Weil's disease, which has
flu-like symptoms and can be fatal.  Again the chances of actually catching it
are tiny.  If you would like more information about these risks, see the
Appendix to this document -- we strongly advise reading it.

All trips have a ``call out time'' using standard procedures ensuring other
reliable parties are aware of who is on the trip, where we are and when we are
expecting to be back.

\subsection*{Cave conservation}

Obviously leaving litter on the way to, from or in the cave ruins the
experience for others, so don't.

Some caves have been heavily used, nevertheless caves may
still have beautiful formations in them which may range from mud deposits to
limestone formations such as stalactites / stalagmites, curtains and straws.
Be very careful with these as they take a very great number of years to form, and can
become permanently damaged if they come in contact with mud or the oils on your
skin.

Remember to watch where you put your head when standing up, and
point out pretties to other people, both to protect and to look at them.  Try
your very hardest not to break them or put muddy hands on them.

\subsection*{Personal details and information}

All novices must have signed that they have read and understood the contents of
this document.

As an additional condition of becoming a member or temporary member of CUCC, the
information requested on the last page of this document must be completed.

Agreement must be made for that information and information regarding financial
transactions between yourself and the Club to be stored on paper and/or
computer by officers of the Club, for the information on last page of this
document to be passed on to insurers, and for financial records to be passed on
to the University of Cambridge if requested or when necessary, and for your financial
details to be passed onto other members of CUCC where required for accounting
purposes.

\newpage
\subsection*{Contacts}

We can't expect to cover everything in this document, but hopefully it will
have also have convinced you that caving is exciting and adventurous and that
you can expect to have a great time.

For more information (any questions,
however bizarre are welcome) contact any of the following people:

\begin{tabular}{lll}
Aaron Curtis & {\tt ac511@cam.ac.uk} & President\\
Serena Povia & {\tt sp422@cam.ac.uk} & Junior Treasurer\\
John Billings & {\tt jnb26@cam.ac.uk} & Secretary and Webmaster\\
Adam Kessler & {\tt apk27@cam.ac.uk} & Social Secretary\\
Lucy Freem & {\tt ljf41@cam.ac.uk} & Meets Secretary\\
Ollie Stevens & {\tt oacs2@cam.ac.uk} & Tackle Master\\
Pete Harley and Olly Madge & {\tt pjrh2@cam.ac.uk} & Training Officers\\
Kathryn Hopkins & {\tt kh311@cam.ac.uk} & Librarian
\end{tabular}

\newpage
\section*{Appendix}

\subsection*{Hypothermia}

Hypothermia is the name of the condition that occurs when your core body
temperature falls below ${\textrm{35}}^{\circ}$C. It is a serious condition that can be fatal.

In any outdoor activity in which you can get wet and/or cold it is important to
know and to recognise the signs of hypothermia and act accordingly to combat
them. Caves, especially in the UK can be both cold and wet and as you may be a
few hours into a long trip, knowing the signs of hypothermia is particularly
important.

Whilst caving you always work together in a team and it is just as important
for you to recognise the signs of hypothermia in other cavers, especially those
less experienced than yourself, as well as noticing should you develop the
symptoms as well.

The symptoms of hypothermia are: shivering (normally only in the early stages),
cold, clammy and pale skin, tiredness, confusion, difficult and slow speech, a
slow faint pulse, slow breathing and lowered levels of responsiveness
(eventually leading to unconsciousness then death). Not all casualties who have
hypothermia shiver, especially in the later stages, as the body only uses
shivering to prevent mild heat loss.

There are several things you can do to combat hypothermia. Understandably a lot
of this advice won't be viable whilst underground as the necessary resources
won't be available, but the information is useful both when out of the cave and
for treating hypothermia in general.

For a mobile casualty who is becoming cold, try to keep them dry and dressed in
warm, dry clothing whenever possible. Give them extra clothing to wear if it is
available and remove wet clothing if possible. Evacuate them from the cave to
warmth and safety as soon as possible.

Reheating should be by gradual rewarming through extra clothing and indirect
heat, rather than by direct means such as rubbing of the skin or a hot water
bottle. This is because direct heat sources near the skin draw blood vessels
towards the skin leading to potential further heat loss, a drop in blood
pressure and putting extra strain on the heart. It is important that the chest
and upper abdomen are kept warm, as well as the head.

For an injured or extremely cold casualty: {\bf Cave Rescue} should be called
immediately by dialling 999 and asking for the Police and explaining that you
require Cave Rescue.   The casualty should be covered in extra layers of dry
clothing and placed in a survival bag. It is particularly important to ensure
that warm layers (e.g.  clothing, a blanket or a spare survival bag) are placed
between the casualty and the ground, as much heat loss is through conduction
with the cold ground.  The casualties' whole body should be covered including
the head.  Continual talking giving reassurance to the casualty always helps to
keep them positive and responsive. It also provides you with a means of
constantly assessing a casualty.

\subsection*{Weil's Disease (Leptospirosis)}

Leptospirosis is an infection caused by the {\em leptospira} bacterium, which
if left untreated can cause severe damage to body organs and jaundice. When it
reaches this late and severe stage it is commonly known as ``Weil's Disease''. Of
those who contract the infection each year, nearly all completely recover with
treatment. The infection is often carried by animals (including cattle) and
rodents (especially rats), and is excreted in their urine into waterways. The
infection can be caught through a whole variety of activities in which you may
come into substances infected by the bacterium.

In caving, it is important to know the symptoms and the seriousness of Weil's
Disease {\em even though the chances of catching it are tiny}, as treatment for any
infection must be caught early.  It is especially important if you have been
caving in an area where caves are in close proximity to farmyards and the like.

The leptospirosis infection resembles a cold or influenza (flu) infection in
the initial stage and has an incubation period of 4 to 10 days with symptoms
lasting for up to 3 weeks. The early symptoms are: fever, chills, muscular
aches and pains, loss of appetite and nausea when lying down. {\bf These can easily
be mistaken for influenza, meningitis or a common fever of unknown origin.} The
fever lasts for approximately five days then significant deterioration in the
later stages occurs.

Do not be afraid to go to a doctor if you are concerned about early possible
symptoms, and impress upon them that you may have been exposed to the bacterium.

The symptoms of the later stages of the infection may also include: bruising of
the skin, anaemia, sore eyes, nosebleeds and jaundice.

IF YOU CONTRACT ANY OF THESE LATER STAGE SYMPTOMS AFTER HAVING WHAT WAS
BELIEVED TO BE FLU OR A COMMON COLD, THEN YOU MUST GO TO YOUR LOCAL HOSPITAL'S
ACCIDENT AND EMERGENCY DEPARTMENT STRAIGHT AWAY AND TELL THEM YOU COULD HAVE
LEPTOSPIROSIS. IF LEFT UNTREATED AT THIS POINT IT COULD BE FATAL.

Several steps can be taken to lower the risk of infection. Firstly, when above
ground, avoid whenever possible swimming or wading through any stagnant pools
of water. Secondly, footwear should be worn at all times.  Finally, both above
ground and whilst caving, all cuts and grazes should be covered by waterproof
plasters. 

\newpage
\input{form.tex}
\end{document}

